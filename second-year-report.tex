\documentclass[a4paper,10pt]{article}

\bibliographystyle{plain}

\usepackage{xspace}
\usepackage{todonotes}
\usepackage{cleveref}
\usepackage{amsmath}
\usepackage{amsthm}
\usepackage{amssymb}
\usepackage{stmaryrd}
\usepackage{url}
\usepackage{mathrsfs}
\usepackage[pdftex, all]{xy}

\title{Second year report}

\author{Gabe Dijkstra}

\newcommand{\todoi}[1]{\todo[inline]{#1}}

\newcommand{\ie}{i.e.\xspace}
\newcommand{\eg}{e.g.\xspace}

\newcommand{\uip}{uniqueness of identity proofs\xspace}
\newcommand{\Uip}{Uniqueness of identity proofs\xspace}
\newcommand{\hott}{homotopy type theory\xspace}
\newcommand{\Hott}{Homotopy type theory\xspace}
\newcommand{\mltt}{Martin-L\"of's type theory\xspace}

\newcommand{\hit}{higher inductive type\xspace}
\newcommand{\Hit}{Higher inductive type\xspace}
\newcommand{\hits}{higher inductive types\xspace}
\newcommand{\Hits}{Higher inductive types\xspace}
\newcommand{\wtypes}{$W$-types\xspace}
\newcommand{\wtype}{$W$-type\xspace}
\newcommand{\oit}{ordinary inductive type\xspace}
\newcommand{\Oit}{Ordinary inductive type\xspace}
\newcommand{\oits}{ordinary inductive types\xspace}
\newcommand{\Oits}{Ordinary inductive types\xspace}

\newcommand{\onehit}{$1$-HIT\xspace}
\newcommand{\onehits}{$1$-HITs\xspace}
\newcommand{\twohit}{$2$-HIT\xspace}
\newcommand{\twohits}{$2$-HITs\xspace}
\newcommand{\nhit}[1]{$#1$-HIT\xspace}
\newcommand{\nhits}[1]{$#1$-HITs\xspace}

\newcommand{\ronehit}{restricted \onehit}
\newcommand{\ronehits}{restricted \onehits}
\newcommand{\Ronehit}{Restricted \onehit}
\newcommand{\Ronehits}{Restricted \onehits}

\newcommand{\zeroconstructor}{$0$-constructor\xspace}
\newcommand{\zeroconstructors}{$0$-constructors\xspace}
\newcommand{\oneconstructor}{$1$-constructor\xspace}
\newcommand{\oneconstructors}{$1$-constructors\xspace}
\newcommand{\twoconstructor}{$2$-constructor\xspace}
\newcommand{\twoconstructors}{$2$-constructors\xspace}
\newcommand{\nconstructor}[1]{$#1$-constructor\xspace}
\newcommand{\nconstructors}[1]{$#1$-constructors\xspace}

\newcommand{\data}{{\bf data}}
\newcommand{\where}{{\bf where}}

\newcommand{\Type}{\mathsf{Type}}

\newcommand{\Nat}{\mathbb{N}}
\newcommand{\indnat}{\mathsf{ind-nat}}
\newcommand{\zero}{\mathsf{zero}}
\newcommand{\suc}{\mathsf{succ}}

\newcommand{\constr}[1]{#1}
\newcommand{\alg}[1]{\mathcal{#1}}
\newcommand{\algcat}[1]{#1\mbox{-alg}}
\newcommand{\Algcat}[1]{#1\mbox{-Alg}}
\newcommand{\cat}[1]{\mathcal{#1}}
\newcommand{\Cat}{\mathsf{Cat}}
\newcommand{\Id}[1]{\mathsf{Id}_{#1}}
\newcommand{\id}[1]{\mathsf{id}_{#1}}
\DeclareMathOperator*{\ido}{id_0}
\DeclareMathOperator*{\idi}{id_1}

\newcommand{\ddefeq}{:\equiv}
\newcommand{\defeq}{\equiv}

\newcommand{\cont}[2]{#1 \lhd #2}
\newcommand{\cext}[2]{\llbracket \cont{#1}{#2} \rrbracket}

\newcommand{\C}{\cat{C}}
\newcommand{\D}{\cat{D}}
\newcommand{\assoc}{\mathsf{assoc}}
\newcommand{\map}{\mathsf{map}}
\newcommand{\ap}{\mathsf{ap}}
\newcommand{\rec}{\mathsf{rec}}
\newcommand{\ind}{\mathsf{ind}}
\newcommand{\inn}{\mathsf{in}}
\newcommand{\funext}{\mathsf{funext}}
\newcommand{\Spec}{\mathsf{Spec}}
\newcommand{\Alg}{\mathsf{Alg}}
\newcommand{\Constr}{\mathsf{Constr}}
\newcommand{\extend}{\mathsf{extend}}

\newcommand{\base}{\mathsf{base}}
\newcommand{\loopcstr}{\mathsf{loop}}
\newcommand{\Eq}{\mathsf{Eq}}
\newcommand{\eq}{\mathsf{eq}}
\newcommand{\catcat}{\mathsf{Cat}}
\newcommand{\trunc}{\mathsf{trunc}}

\newtheorem{prop}{Proposition}[section]
\newtheorem{thm}[prop]{Theorem}
\newtheorem{lem}[prop]{Lemma}
\newtheorem{cor}[prop]{Corollary}
\newtheorem{conjecture}[prop]{Conjecture}
\theoremstyle{definition}
\newtheorem{defn}[prop]{Definition}
\theoremstyle{remark}

\crefname{prop}{proposition}{propositions}
\crefname{thm}{theorem}{theorems}
\crefname{lem}{lemma}{lemmas}
\crefname{cor}{corollary}{corollaries}
\crefname{defn}{definition}{definitions}
\crefname{equation}{\!}{\!}

\DeclareMathOperator{\isinitial}{\mathsf{is-initial}}
\DeclareMathOperator{\iscontr}{\mathsf{is-contr}}
\DeclareMathOperator{\total}{\mathsf{total}}
\DeclareMathOperator{\proj}{\mathsf{proj}}
\DeclareMathOperator{\refl}{\mathsf{refl}}
\DeclareMathOperator{\lift}{\mathsf{lift}}
\DeclareMathOperator{\boolnot}{\mathsf{not}}
\DeclareMathOperator{\Bool}{\mathsf{Bool}}
\DeclareMathOperator{\quot}{\mathsf{quot}}
\DeclareMathOperator{\uipc}{\mathsf{uip}}
\DeclareMathOperator{\Tree}{\mathsf{Tree}}
\DeclareMathOperator{\Pretree}{\mathsf{Pretree}}
\DeclareMathOperator{\coe}{\mathsf{coe}}
\DeclareMathOperator{\Ctx}{\mathsf{Ctx}}
\DeclareMathOperator{\Ty}{\mathsf{Ty}}
\DeclareMathOperator{\snoc}{\mathsf{snoc}}
\DeclareMathOperator{\uncurry}{\mathsf{uncurry}}
\DeclareMathOperator{\leftid}{\mathsf{left-id}}


\newcommand{\Fam}{\mathsf{Fam}}
\newcommand{\Famover}[2]{\Fam_{#1}\ #2}
\newcommand{\fib}{\mathsf{Fib}}
\newcommand{\fibover}[2]{\fib_{#1}\ #2}
\newcommand{\algfam}[1]{\mathcal{#1}}
\newcommand{\algfib}[1]{\mathcal{#1}}
\newcommand{\sctn}{\mathsf{Sect}}
\newcommand{\sectionof}[3]{\sctn_{#1}\ #2\ #3}
\newcommand{\actionsection}[1]{\bar{#1}}

\newcommand{\inverseImage}[1]{\mathsf{fib}_{#1}}

% Path composition from The Book.
\newcommand{\ct}{%
  \mathchoice{\mathbin{\raisebox{0.5ex}{$\displaystyle\centerdot$}}}%
             {\mathbin{\raisebox{0.5ex}{$\centerdot$}}}%
             {\mathbin{\raisebox{0.25ex}{$\scriptstyle\,\centerdot\,$}}}%
             {\mathbin{\raisebox{0.1ex}{$\scriptscriptstyle\,\centerdot\,$}}}
}

\newcommand{\To}{\Rightarrow}
\newcommand{\unitty}{\top}
\newcommand{\toruswrong}{T^2\mbox{-}\mathsf{wrong}}
\newcommand{\Fid}{F\mbox{-}\mathsf{id}}
\newcommand{\Fcomp}{F\mbox{-}\circ}
\newcommand{\Gid}{G\mbox{-}\mathsf{id}}
\newcommand{\Gcomp}{G\mbox{-}\circ}
\newcommand{\Gtimes}{G\mbox{-}\times}
\newcommand{\Geq}{G\mbox{-}\mathsf{eq}}
\newcommand{\reasontext}[1]{\{\ \textsf{#1}\ \}}
\newcommand{\reasonterm}[1]{\{\ #1\ \}}
\newcommand{\initialfield}{\mathcal{F}}
\newcommand{\omegacat}{$(\infty,1)$-category}
\newcommand{\omegacats}{$(\infty,1)$-categories}


\begin{document}

\maketitle

test~\cite{UFP2013}.

\section{Introduction}
\label{sec:introduction}

\section{Syntax of higher inductive types}
\label{sec:syntax}

If we define an ordinary inductive type, we start out by writing down
a list of constructors, \eg:
%
\begin{alignat*}{2}
  &\rlap{$\data\ T\ :\ \Type\ \where$} \\
  &&\ \ c_0  &:\ F_0\ T \to T \\
  &&\ \ c_1  &:\ F_1\ T \to T \\
  &&\ \      &\vdots \\
  &&\ \ c_k  &:\ F_k\ T \to T
\end{alignat*}
%
where $F_i : \Type \to \Type$. We require each $F_i$ to be a (strictly
positive) functor, in order to make sense of the computation rules of
the elimination operator. A constructor $c_i$ is then an algebra of
functor $F_i$ with carrier $T$. Equivalently, we can define an
inductive type with a single constructor:
%
\begin{alignat*}{2}
  &\rlap{$\data\ T\ :\ \Type\ \where$} \\
  &&\ \ c &:\ F\ T \to T 
\end{alignat*}
%
where $F\ X \ddefeq F_0\ X + F_1\ X + \hdots + F_k\ X$, so a single
functor is all we need to describe an \oit.

In the case of \hits, the situation is more involved. Consider for
example the circle data type:
%
\begin{alignat*}{2}
  &\rlap{$\data\ S^1\ :\ \Type\ \where$} \\
  &&\ \ \base     &:\ S^1 \\
  &&\ \ \loopcstr &:\ \base = \base
\end{alignat*}

There are two things that are different from our previous
situation. Firstly, the result type of $\loopcstr$ is not $S^1$, but a path
space of $S^1$: constructors are no longer algebras of functor, but a
kind of \emph{dialgebra}: the arguments as well as the result type of
a constructor may vary. Secondly, the $\loopcstr$ constructor refers to the
previous constructor $\base$.

The result type of a constructor can also depend on the value of its
arguments, as we see in the definition of propositional truncation as
a \hit:
%
\begin{alignat*}{2}
  &\rlap{$\data\ || A ||\ :\ \Type \where$} \\
  &&\ \ [\_]   &:\ A \to || A || \\
  &&\ \ \trunc &:\ (x\ y : || A ||) \to x = y
\end{alignat*}

Constructors of a \hit are \emph{dependent dialgebras}. In general, a
higher inductive type looks as follows:
%
\begin{alignat*}{5}
  &\rlap{$\data\ T\ :\ \Type\ \where$}& \\
  &&\ \ c_0    &: (x : F_0 T)                    &\to\ &G_0\ (T, x)& \\
  &&\ \ c_1    &: (x : F_1 (T,c_0))              &\to\ &G_1\ ((T, c_0), x)& \\
  &&\ \ c_2    &: (x : F_2 (T,c_0,c_1))          &\to\ &G_2\ ((T, c_0, c_1), x)& \\
  &&\ \        &\vdots &&\\
  &&\ \ c_{k+1} & : (x : F_k (T,c_0, \hdots, c_k) &\to\ &G_k\ ((T, c_0, \hdots, c_k), x)&
\end{alignat*}

We will refer to the $F_i$ functors as \emph{argument} functors and the
$G_i$ functors as \emph{target} functors. The types of the argument
functors are:
%
\begin{alignat*}{6}
&F_0     &\ :\ &\Type                &\ \to \Type& \\
&F_1     &\ :\ &\algcat{(F_0, G_0)}  &\ \to \Type& \\      
&F_2     &\ :\ &\algcat{(F_1, G_1)}  &\ \to \Type& \\      
&F_{k+1} &\ :\ &\algcat{(F_k, G_k)}  &\ \to \Type&
\end{alignat*}
%
where $\algcat{(F0, G0)}$ is the category whose objects are dependent
dialgebras $(X : \Type) \times (\theta : (x : F_0\ X) \to G_0\ (X,\ x))$. The category
$\algcat{(F_{i+1},G_{i+1})}$ has as objects: $(X : \algcat{(F_i,G_i)}) \times (\theta : (x :
F_{i+1}\ X) \to G_{i+1}\ (X, x)$. The target functors also take the value of
the arguments as an argument, so they have the following types:
%
\begin{alignat*}{6}
&G_0     &\ :\ &\int_{\Type} F_0                &\ \to \Type& \\
&G_1     &\ :\ &\int_{\algcat{(F_0, G_0)}} F_1  &\ \to \Type& \\      
&G_2     &\ :\ &\int_{\algcat{(F_1, G_1)}} F_2  &\ \to \Type& \\      
&G_{k+1} &\ :\ &\int_{\algcat{(F_k, G_k)}} F_{k+1}  &\ \to \Type&
\end{alignat*}

We see that the general shape of a constructor is a dependent dialgebra:
$$
c : (x : F\ X) \to G\ (X , x)
$$
where $\C : \catcat$, $F : \C \to \Type$ and
$G : \int_{\C} F \to \Type$, where $\C$ is either $\Type$ or some
category of dependent dialgebras, ``containing'' all the previous
constructors.

When describing \hits, we do not allow for any target functor $G$: it
must either return the type we are defining or a (possibly iterated)
path space of that type. A constructor of which the target is an
$n$-times iterated path space is called an \emph{$n$-constructor}. The
$0$-times iterated path space of a type $A$ is simply the type $A$
itself, hence ordinary constructors are \zeroconstructors. We call a
\hit an $n$-\hit if $n$ is the maximum iteration depth of its
constructors.

A \emph{\zeroconstructor} or \emph{point constructor} is a dialgebra:
$$
c : (x : F\ X) \to U\ X
$$
where $\C : \catcat$, $F : \C \to \Type$ and $U : \C \to Type$ its forgetful
functor.

A \emph{\oneconstructor} is a dependent dialgebra of which the target
functor returns a path space:
$$
  c : (x : F\ X) \to \Eq_0\ (X, x)
$$
where $\C : \catcat$, $F, U : \C \to \Type$, and $\Eq_0 : \int_{\C} F \to
\Type$ is the functor:
$$
  \Eq_0\ (X , x) \ddefeq (l_0\ x = r_0\ x)
$$
where $l_0,\ r_0 : F \to U$ are natural transformations.

For higher path constructors, we have to specify a tower of natural
transformations, specifying the end points at every level. To specify
an $(k+1)$-constructor, we need to give the natural transformations:
%
\begin{alignat*}{5}
  &l_0,\ r_0  &\ :\ &F \to U& \\
  &l_1,\ r_1  &\ : \ &1 \to \Eq_0& \\
  &l_2,\ r_2  &\ : \ &1 \to \Eq_1& \\
  &           &\vdots&& \\
  &l_k,\ r_k  &\ : \ &1 \to \Eq_{k-1}&
\end{alignat*}
%
where $\Eq_i\ (X, x) \ddefeq (l_i\ x = r_i\ x)$. $Eq_n$ is then the target
functor we are interested in.

Even when we consider only \zeroconstructors, the above scheme is more
general than the schemes that are usually considered when talking
about \oits: any constructor may refer to any previous
constructor, \eg:
%
\begin{alignat*}{5}
  &\rlap{$\data\ T\ :\ \Type\ \where$} \\
  &\ \ c_0 &\ :\ &T&\\
  &\ \ c_1 &\ :\ &c_0 = c_0 \to T&
\end{alignat*}
%
In fact, Agda even allows such definitions.


\section{Initiality and induction for \oits}
\label{sec:oits}

As mentioned in~\cref{sec:syntax}, an \oit is fully described by a
functor $F : \Type \to \Type$.

\subsection{Induction principle}
\label{sec:oitinduction}

The induction principle of an inductive type $T$ gives us a method to
show that for a predicate $P : T \to \Type$, for all $x : T$, $P\ x$
holds. Intuitively we have to show for each constructor $c$ that if
$P$ holds for all $T$-subterms of $x$, then $P$ holds for $c x$. For
example, for the natural numbers, we have the induction operator:

\todoi{elim nat}

The induction operator also comes with computation rules:

\todoi{elim nat comp rules}

Generalising this to an arbitrary inductive type $T$ given by a
functor $F : \Type \to \Type$, we need to be able to state that a
predicate $P : T \to \Type$ holds for all $T$-subterms of $x : FT$. We
will do this by introducing a modality
$\Box_F : (A \to \Type) \to FA \to \Type$, which is defined by the
following equation, for $A : \Type$, $B : A \to \Type$:
$$
F (\Sigma\ A\ B) = \Sigma\ FA\ (\Box_{F} B)
$$
By singleton elimination we can then give the following definition of $\Box_{F}\ B$:
$$
\Box_{F}\ B\ x = (y : F (\Sigma\ A\ B)) \times (F\ \pi_1\ x = y)
$$
The induction principle for $T$ then becomes:

\todoi{elim T}

with computation rules:

\todoi{elim T comp rules and introduce $\actionsection{F}$?}

Usually $P$ is called the \emph{motive} of the induction and $m$ its
\emph{method}. One can also think of a motive along with a method as a
\emph{family} of algebras:

\begin{defn}
  An \emph{$F$-algebra family} over an algebra $(X,\theta)$, denoted
  $\Famover{\algcat{F}}{(X,\theta)}$, consists of:

  \begin{itemize}
  \item $P : X \to \Type$
  \item $m : (x : F\ X) \to \Box_F\ P\ x \to P\ (\theta\ x)$
  \end{itemize}
\end{defn}

In $\Type$, giving a family over a type $A : \Type$ is the same as
giving a function into $A$, \ie:
$$
(A \to \Type) = (B : \Type) \times (p : B \to A)
$$
In~\cref{sec:oitsectioninduction} we will show that giving a
family over an algebra $\alg{X} : \algcat{F}$ is the same as giving an
algebra morphism into $\alg{X}$:
$$
\Famover{\algcat{F}}{\alg{X}} = (\alg{Y} : \algcat{F}) \times \algcat{F}(\alg{Y},\alg{X})
$$

\subsection{Initiality}
\label{sec:oitinitiality}

\subsection{Section principle}
\label{sec:oitsection}

Instead of showing directly that the induction principle is logically
equivalent to homotopy initiality, we introduce another principle that
is somewhere in between the type theoretic induction principle and the
categorical notion of initiality. In~\cref{sec:oitinduction}, the
induction principle for an inductive type $T$ gives us for every
algebra family $(P,m)$, a dependent function $(x : T) \to P\ x$
subject to certain computation rules. The section principle states
that every \emph{algebra fibration} has a \emph{section}.

\begin{defn}
  An \emph{$F$-algebra fibration} over an algebra $\alg{X}$ is an
  algebra $\alg{Y}$ along with an algebra morphism
  $\algcat{F}(\alg{Y},\alg{X})$. The type of $F$-algebra fibrations
  over $\alg{X}$ is denoted as $\fibover{\algcat{F}}{\alg{X}}$
\end{defn}

\begin{defn}
  Given $\alg{X} : \algcat{F}$ and $\algfib{P} : \fibover{\algcat{F}}{\alg{X}}$,
\end{defn}


\subsection{Induction principle is logically equivalent to section principle}
\label{sec:oitsectioninduction}

\subsection{Section principle is logically equivalent to initiality}
\label{sec:oitsectioninitiality}

We will first show that initiality implies the section induction
principle. Suppose $\alg{X} : \algcat{F}$ is initial,
$\alg{Y} : \algcat{F}$ and $p : \algcat{F}(\alg{Y},\alg{X})$. Since
$\alg{X}$ is initial, there is a morphism
$s : \algcat{F}(\alg{X},\alg{Y})$. Since $\alg{X}$ is initial, there
is only one morphism $\algcat{F}(\alg{X},\alg{X})$, namely
$\Id{\alg{X}}$, hence $p \circ s = \Id{\alg{X}}$, therefore $s$ is a
section of $p$ and we are done.

Conversely, let $\alg{X} : \algcat{F}$ be an algebra that satisfies
the section principle, then we have to show that for any algebra
$Y : \algcat{F}$ there exists a unique algebra morphism
$f : \algcat{F}(\alg{X},\alg{Y})$. Given $\alg{X}$ and $\alg{Y}$, we
can form the product of algebras $\alg{X} \times \alg{Y}$, which comes
with projections $\pi_1 : \algcat{F}(\alg{X} \times \alg{Y}, \alg{X})$
and $\pi_2 : \algcat{F}(\alg{X} \times \alg{Y}, \alg{Y})$. Let
$s : \algcat{F}(\alg{X},\alg{X} \times \alg{Y})$ be the section of
$\pi_1$. We can then construct the $f$ as follows:
$$
\xymatrix{
  &\alg{X} \ar[r]^-{s} &\alg{X} \times \alg{Y} \ar[r]^-{\pi_2} &\alg{Y}
}
$$
Now we have to show that for any $g : \algcat{F}(\alg{X},\alg{Y})$,
  $f = g$. This can be done by constructing the equaliser $\alg{E}$ of
  $f$ and $g$:
$$
\xymatrix{
  &\alg{E} \ar[r]^{e} &\alg{X} \ar@<-.5ex>[r]_-{g} \ar@<.5ex>[r]^-{f} &\alg{Y}
}
$$
We get a section of $e$, which then shows that $f = g$.

\subsection{Category of algebras}
\label{sec:oitalgcat}


\section{Initiality and induction for \hits}
\label{sec:hits}



\bibliography{second-year-report}

\end{document}

