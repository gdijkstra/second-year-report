\documentclass[a4paper,10pt]{article}

\usepackage{color}
\usepackage[usenames,dvipsnames,svgnames,table]{xcolor}


%% Encoding, font stuff
\usepackage[utf8x]{inputenc}
\usepackage{autofe}
\usepackage{ucs}
\usepackage[greek,english]{babel}
\usepackage{bbm}

%% Symbols and whatever
\usepackage{amsmath,amsthm,amssymb,amsfonts,stmaryrd}

% Format file based on agda.fmt as included in lhs2TeX distribution.

%% ODER: format ==         = "\mathrel{==}"
%% ODER: format /=         = "\neq "
%
%
\makeatletter
\@ifundefined{lhs2tex.lhs2tex.sty.read}%
  {\@namedef{lhs2tex.lhs2tex.sty.read}{}%
   \newcommand\SkipToFmtEnd{}%
   \newcommand\EndFmtInput{}%
   \long\def\SkipToFmtEnd#1\EndFmtInput{}%
  }\SkipToFmtEnd

\newcommand\ReadOnlyOnce[1]{\@ifundefined{#1}{\@namedef{#1}{}}\SkipToFmtEnd}
\usepackage{amstext}
\usepackage{amssymb}
\usepackage{stmaryrd}
\DeclareFontFamily{OT1}{cmtex}{}
\DeclareFontShape{OT1}{cmtex}{m}{n}
  {<5><6><7><8>cmtex8
   <9>cmtex9
   <10><10.95><12><14.4><17.28><20.74><24.88>cmtex10}{}
\DeclareFontShape{OT1}{cmtex}{m}{it}
  {<-> ssub * cmtt/m/it}{}
\newcommand{\texfamily}{\fontfamily{cmtex}\selectfont}
\DeclareFontShape{OT1}{cmtt}{bx}{n}
  {<5><6><7><8>cmtt8
   <9>cmbtt9
   <10><10.95><12><14.4><17.28><20.74><24.88>cmbtt10}{}
\DeclareFontShape{OT1}{cmtex}{bx}{n}
  {<-> ssub * cmtt/bx/n}{}
\newcommand{\tex}[1]{\text{\texfamily#1}}	% NEU

\newcommand{\Sp}{\hskip.33334em\relax}


\newcommand{\Conid}[1]{\mathit{#1}}
\newcommand{\Varid}[1]{\mathit{#1}}
\newcommand{\anonymous}{\kern0.06em \vbox{\hrule\@width.5em}}
\newcommand{\plus}{\mathbin{+\!\!\!+}}
\newcommand{\bind}{\mathbin{>\!\!\!>\mkern-6.7mu=}}
\newcommand{\rbind}{\mathbin{=\mkern-6.7mu<\!\!\!<}}% suggested by Neil Mitchell
\newcommand{\sequ}{\mathbin{>\!\!\!>}}
\renewcommand{\leq}{\leqslant}
\renewcommand{\geq}{\geqslant}
\usepackage{polytable}

%mathindent has to be defined
\@ifundefined{mathindent}%
  {\newdimen\mathindent\mathindent\leftmargini}%
  {}%

\def\resethooks{%
  \global\let\SaveRestoreHook\empty
  \global\let\ColumnHook\empty}
\newcommand*{\savecolumns}[1][default]%
  {\g@addto@macro\SaveRestoreHook{\savecolumns[#1]}}
\newcommand*{\restorecolumns}[1][default]%
  {\g@addto@macro\SaveRestoreHook{\restorecolumns[#1]}}
\newcommand*{\aligncolumn}[2]%
  {\g@addto@macro\ColumnHook{\column{#1}{#2}}}

\resethooks

\newcommand{\onelinecommentchars}{\quad-{}- }
\newcommand{\commentbeginchars}{\enskip\{-}
\newcommand{\commentendchars}{-\}\enskip}

\newcommand{\visiblecomments}{%
  \let\onelinecomment=\onelinecommentchars
  \let\commentbegin=\commentbeginchars
  \let\commentend=\commentendchars}

\newcommand{\invisiblecomments}{%
  \let\onelinecomment=\empty
  \let\commentbegin=\empty
  \let\commentend=\empty}

\visiblecomments

\newlength{\blanklineskip}
\setlength{\blanklineskip}{0.66084ex}

\newcommand{\hsindent}[1]{\quad}% default is fixed indentation
\let\hspre\empty
\let\hspost\empty
\newcommand{\NB}{\textbf{NB}}
\newcommand{\Todo}[1]{$\langle$\textbf{To do:}~#1$\rangle$}

\EndFmtInput
\makeatother
%
%
%
%
%
%
% This package provides two environments suitable to take the place
% of hscode, called "plainhscode" and "arrayhscode". 
%
% The plain environment surrounds each code block by vertical space,
% and it uses \abovedisplayskip and \belowdisplayskip to get spacing
% similar to formulas. Note that if these dimensions are changed,
% the spacing around displayed math formulas changes as well.
% All code is indented using \leftskip.
%
% Changed 19.08.2004 to reflect changes in colorcode. Should work with
% CodeGroup.sty.
%
\ReadOnlyOnce{polycode.fmt}%
\makeatletter

\newcommand{\hsnewpar}[1]%
  {{\parskip=0pt\parindent=0pt\par\vskip #1\noindent}}

% can be used, for instance, to redefine the code size, by setting the
% command to \small or something alike
\newcommand{\hscodestyle}{}

% The command \sethscode can be used to switch the code formatting
% behaviour by mapping the hscode environment in the subst directive
% to a new LaTeX environment.

\newcommand{\sethscode}[1]%
  {\expandafter\let\expandafter\hscode\csname #1\endcsname
   \expandafter\let\expandafter\endhscode\csname end#1\endcsname}

% "compatibility" mode restores the non-polycode.fmt layout.

\newenvironment{compathscode}%
  {\par\noindent
   \advance\leftskip\mathindent
   \hscodestyle
   \let\\=\@normalcr
   \let\hspre\(\let\hspost\)%
   \pboxed}%
  {\endpboxed\)%
   \par\noindent
   \ignorespacesafterend}

\newcommand{\compaths}{\sethscode{compathscode}}

% "plain" mode is the proposed default.
% It should now work with \centering.
% This required some changes. The old version
% is still available for reference as oldplainhscode.

\newenvironment{plainhscode}%
  {\hsnewpar\abovedisplayskip
   \advance\leftskip\mathindent
   \hscodestyle
   \let\hspre\(\let\hspost\)%
   \pboxed}%
  {\endpboxed%
   \hsnewpar\belowdisplayskip
   \ignorespacesafterend}

\newenvironment{oldplainhscode}%
  {\hsnewpar\abovedisplayskip
   \advance\leftskip\mathindent
   \hscodestyle
   \let\\=\@normalcr
   \(\pboxed}%
  {\endpboxed\)%
   \hsnewpar\belowdisplayskip
   \ignorespacesafterend}

% Here, we make plainhscode the default environment.

\newcommand{\plainhs}{\sethscode{plainhscode}}
\newcommand{\oldplainhs}{\sethscode{oldplainhscode}}
\plainhs

% The arrayhscode is like plain, but makes use of polytable's
% parray environment which disallows page breaks in code blocks.

\newenvironment{arrayhscode}%
  {\hsnewpar\abovedisplayskip
   \advance\leftskip\mathindent
   \hscodestyle
   \let\\=\@normalcr
   \(\parray}%
  {\endparray\)%
   \hsnewpar\belowdisplayskip
   \ignorespacesafterend}

\newcommand{\arrayhs}{\sethscode{arrayhscode}}

% The mathhscode environment also makes use of polytable's parray 
% environment. It is supposed to be used only inside math mode 
% (I used it to typeset the type rules in my thesis).

\newenvironment{mathhscode}%
  {\parray}{\endparray}

\newcommand{\mathhs}{\sethscode{mathhscode}}

% texths is similar to mathhs, but works in text mode.

\newenvironment{texthscode}%
  {\(\parray}{\endparray\)}

\newcommand{\texths}{\sethscode{texthscode}}

% The framed environment places code in a framed box.

\def\codeframewidth{\arrayrulewidth}
\RequirePackage{calc}

\newenvironment{framedhscode}%
  {\parskip=\abovedisplayskip\par\noindent
   \hscodestyle
   \arrayrulewidth=\codeframewidth
   \tabular{@{}|p{\linewidth-2\arraycolsep-2\arrayrulewidth-2pt}|@{}}%
   \hline\framedhslinecorrect\\{-1.5ex}%
   \let\endoflinesave=\\
   \let\\=\@normalcr
   \(\pboxed}%
  {\endpboxed\)%
   \framedhslinecorrect\endoflinesave{.5ex}\hline
   \endtabular
   \parskip=\belowdisplayskip\par\noindent
   \ignorespacesafterend}

\newcommand{\framedhslinecorrect}[2]%
  {#1[#2]}

\newcommand{\framedhs}{\sethscode{framedhscode}}

% The inlinehscode environment is an experimental environment
% that can be used to typeset displayed code inline.

\newenvironment{inlinehscode}%
  {\(\def\column##1##2{}%
   \let\>\undefined\let\<\undefined\let\\\undefined
   \newcommand\>[1][]{}\newcommand\<[1][]{}\newcommand\\[1][]{}%
   \def\fromto##1##2##3{##3}%
   \def\nextline{}}{\) }%

\newcommand{\inlinehs}{\sethscode{inlinehscode}}

% The joincode environment is a separate environment that
% can be used to surround and thereby connect multiple code
% blocks.

\newenvironment{joincode}%
  {\let\orighscode=\hscode
   \let\origendhscode=\endhscode
   \def\endhscode{\def\hscode{\endgroup\def\@currenvir{hscode}\\}\begingroup}
   %\let\SaveRestoreHook=\empty
   %\let\ColumnHook=\empty
   %\let\resethooks=\empty
   \orighscode\def\hscode{\endgroup\def\@currenvir{hscode}}}%
  {\origendhscode
   \global\let\hscode=\orighscode
   \global\let\endhscode=\origendhscode}%

\makeatother
\EndFmtInput
%
%
\ReadOnlyOnce{note.fmt}%


\providecommand\mathbbm{\mathbb}

% Path composition from The Book.
\newcommand{\ct}{%
  \mathchoice{\mathbin{\raisebox{0.5ex}{$\displaystyle\centerdot$}}}%
             {\mathbin{\raisebox{0.5ex}{$\centerdot$}}}%
             {\mathbin{\raisebox{0.25ex}{$\scriptstyle\,\centerdot\,$}}}%
             {\mathbin{\raisebox{0.1ex}{$\scriptscriptstyle\,\centerdot\,$}}}
}

\DeclareUnicodeCharacter{737}{\textsuperscript{l}}
\DeclareUnicodeCharacter{8718}{\ensuremath{\blacksquare}}
\DeclareUnicodeCharacter{8759}{::}
\DeclareUnicodeCharacter{9669}{\ensuremath{\triangleleft}}
\DeclareUnicodeCharacter{9665}{\ensuremath{\triangleleft}}
\DeclareUnicodeCharacter{8799}{\ensuremath{\stackrel{\scriptscriptstyle ?}{=}}}
\DeclareUnicodeCharacter{10214}{\ensuremath{\llbracket}}
\DeclareUnicodeCharacter{10215}{\ensuremath{\rrbracket}}
\DeclareUnicodeCharacter{946}{\ensuremath{\beta}}
\DeclareUnicodeCharacter{8729}{\ensuremath{\ct}}
% TODO: This is in general not a good idea.
\providecommand\textepsilon{$\epsilon$}
\providecommand\textmu{$\mu$}
\providecommand\textbeta{$\beta$}
\providecommand\texttheta{$\theta$}


%Actually, varsyms should not occur in Agda output.

\renewcommand\Varid[1]{\mathord{\textsf{#1}}}
\let\Conid\Varid
\newcommand\Keyword[1]{\textsf{\textbf{#1}}}
\EndFmtInput



%% Layout stuff and other related goodies.
\usepackage[parfill]{parskip}
\usepackage{xspace}
\usepackage{todonotes}


%% URLs and other link stuff
\usepackage{url}
\usepackage{hyperref}
\usepackage[noabbrev]{cleveref}

\hypersetup{
  colorlinks,
  citecolor=DarkBlue,
  linkcolor=black,
  urlcolor=DarkBlue}

% Bibliography stuff
\usepackage[autostyle]{csquotes}

%% Commands
\newcommand{\todoi}[1]{\todo[inline]{#1}}
\newcommand{\withoutk}{\xspace\text{\tt \char45{}\char45{}without\char45{}K}\xspace}
\newcommand{\uip}{uniqueness of identity proofs\xspace}
\newcommand{\Uip}{Uniqueness of identity proofs\xspace}
\newcommand{\hott}{homotopy type theory\xspace}
\newcommand{\Hott}{Homotopy type theory\xspace}
\newcommand{\mltt}{Martin-L\"of's type theory\xspace}
\newcommand{\hit}{higher inductive type\xspace}
\newcommand{\Hit}{Higher inductive type\xspace}
\newcommand{\hits}{higher inductive types\xspace}
\newcommand{\Hits}{Higher inductive types\xspace}
\newcommand{\oit}{ordinary inductive type\xspace}
\newcommand{\Oit}{Ordinary inductive type\xspace}
\newcommand{\oits}{ordinary inductive types\xspace}
\newcommand{\Oits}{Ordinary inductive types\xspace}
\newcommand{\zeroconstructor}{$0$-constructor\xspace}
\newcommand{\zeroconstructors}{$0$-constructors\xspace}
\newcommand{\onehit}{$1$-HIT\xspace}
\newcommand{\onehits}{$1$-HITs\xspace}
\newcommand{\oneconstructor}{$1$-constructor\xspace}
\newcommand{\oneconstructors}{$1$-constructors\xspace}
\newcommand{\twohit}{$2$-HIT\xspace}
\newcommand{\twohits}{$2$-HITs\xspace}
\newcommand{\twoconstructor}{$2$-constructor\xspace}
\newcommand{\twoconstructors}{$2$-constructors\xspace}
\newcommand{\zerovariable}{$0$-variable\xspace}
\newcommand{\onevariable}{$1$-variable\xspace}
\newcommand{\twovariable}{$2$-variable\xspace}
\newcommand{\zerovariables}{$0$-variables\xspace}
\newcommand{\onevariables}{$1$-variables\xspace}
\newcommand{\twovariables}{$2$-variables\xspace}
\newcommand{\ie}{i.e.\xspace}
\newcommand{\eg}{e.g.\xspace}
\newcommand{\Type}{\mbox{Type}}
\newcommand{\wtypes}{$W$-types\xspace}
\newcommand{\wtype}{$W$-type\xspace}
\newcommand{\ronehit}{restricted \onehit\xspace}
\newcommand{\ronehits}{restricted \onehits\xspace}
\newcommand{\Ronehit}{Restricted \onehit\xspace}
\newcommand{\Ronehits}{Restricted \onehits\xspace}

\title{Notes on \onehits}

\author{Gabe Dijkstra}

\date{}

\begin{document}

\maketitle

\section{Introduction}

\section{Describing \hits}


\section{\Ronehits}



\section{Homotopy initiality for \wtypes}

In type theory, a data type is described by its introduction rule (the
constructors) and its elimination rule (the induction principle) along
with a computation rule describing how these two rules interact.

\todoi{Natural numbers example?}

\subsection{Induction principle}

Given a functor \ensuremath{\Conid{F}\;\mathbin{:}\;\Conid{Type}\;\to \;\Conid{Type}} given as a container \ensuremath{\Conid{F}\;:\equiv\;\Conid{S}\;\Varid{◁}\;\Conid{P}}, the \wtype \ensuremath{\Conid{W}} is defined as having the following
introduction rule / constructor:

\begin{hscode}\SaveRestoreHook
\column{B}{@{}>{\hspre}l<{\hspost}@{}}%
\column{3}{@{}>{\hspre}l<{\hspost}@{}}%
\column{E}{@{}>{\hspre}l<{\hspost}@{}}%
\>[3]{}\Varid{c}\;\mathbin{:}\;\Conid{F}\;\Conid{W}\;\to \;\Conid{W}{}\<[E]%
\ColumnHook
\end{hscode}\resethooks

as well as an elimination rule / induction principle:

\begin{hscode}\SaveRestoreHook
\column{B}{@{}>{\hspre}l<{\hspost}@{}}%
\column{3}{@{}>{\hspre}l<{\hspost}@{}}%
\column{4}{@{}>{\hspre}l<{\hspost}@{}}%
\column{8}{@{}>{\hspre}l<{\hspost}@{}}%
\column{E}{@{}>{\hspre}l<{\hspost}@{}}%
\>[3]{}\Varid{ind}\;\mathbin{:}\;{}\<[E]%
\\
\>[3]{}\hsindent{5}{}\<[8]%
\>[8]{}(\Conid{B}\;\mathbin{:}\;\Conid{W}\;\to \;\Conid{Type})\;{}\<[E]%
\\
\>[3]{}\hsindent{5}{}\<[8]%
\>[8]{}(\Varid{m}\;\mathbin{:}\;(\Varid{x}\;\mathbin{:}\;\Conid{F}\;\Conid{W})\;\Varid{→}\;\Varid{□}\;\Conid{F}\;\Conid{B}\;\Varid{x}\;\Varid{→}\;\Conid{B}\;(\Varid{c}\;\Varid{x}))\;{}\<[E]%
\\
\>[3]{}\hsindent{5}{}\<[8]%
\>[8]{}(\Varid{x}\;\mathbin{:}\;\Conid{W})\;{}\<[E]%
\\
\>[3]{}\hsindent{1}{}\<[4]%
\>[4]{}\to \;{}\<[8]%
\>[8]{}\Conid{B}\;\Varid{x}{}\<[E]%
\ColumnHook
\end{hscode}\resethooks

with computation rule:

\begin{hscode}\SaveRestoreHook
\column{B}{@{}>{\hspre}l<{\hspost}@{}}%
\column{3}{@{}>{\hspre}l<{\hspost}@{}}%
\column{4}{@{}>{\hspre}l<{\hspost}@{}}%
\column{8}{@{}>{\hspre}l<{\hspost}@{}}%
\column{E}{@{}>{\hspre}l<{\hspost}@{}}%
\>[3]{}\Varid{ind-β₀}\;\mathbin{:}\;{}\<[E]%
\\
\>[3]{}\hsindent{5}{}\<[8]%
\>[8]{}(\Conid{B}\;\mathbin{:}\;\Conid{W}\;\to \;\Conid{Type})\;{}\<[E]%
\\
\>[3]{}\hsindent{5}{}\<[8]%
\>[8]{}(\Varid{m}\;\mathbin{:}\;(\Varid{x}\;\mathbin{:}\;\Conid{F}\;\Conid{W})\;\Varid{→}\;\Varid{□}\;\Conid{F}\;\Conid{B}\;\Varid{x}\;\Varid{→}\;\Conid{B}\;(\Varid{c}\;\Varid{x}))\;{}\<[E]%
\\
\>[3]{}\hsindent{5}{}\<[8]%
\>[8]{}(\Varid{x}\;\mathbin{:}\;\Conid{F}\;\Conid{W})\;{}\<[E]%
\\
\>[3]{}\hsindent{1}{}\<[4]%
\>[4]{}\to \;{}\<[8]%
\>[8]{}\Varid{ind}\;\Conid{B}\;\Varid{m}\;(\Varid{c}\;\Varid{x})\;=\;\Varid{m}\;\Varid{x}\;(\Varid{□-lift}\;\Conid{F}\;(\Varid{ind}\;\Conid{B}\;\Varid{m})\;\Varid{x}){}\<[E]%
\ColumnHook
\end{hscode}\resethooks

\subsection{Algebras}

The type of \emph{\ensuremath{\Conid{F}}-algebras}, or simply \emph{algebras}, can be
defined as follows:

\begin{hscode}\SaveRestoreHook
\column{B}{@{}>{\hspre}l<{\hspost}@{}}%
\column{3}{@{}>{\hspre}l<{\hspost}@{}}%
\column{E}{@{}>{\hspre}l<{\hspost}@{}}%
\>[3]{}\Conid{Alg}\;:\equiv\;(\Conid{X}\;\mathbin{:}\;\Conid{Type})\;\Varid{×}\;(\Varid{θ}\;\mathbin{:}\;\Conid{F}\;\Conid{X}\;\to \;\Conid{X}){}\<[E]%
\ColumnHook
\end{hscode}\resethooks

where the type morphisms is defined as follows:

\begin{hscode}\SaveRestoreHook
\column{B}{@{}>{\hspre}l<{\hspost}@{}}%
\column{3}{@{}>{\hspre}l<{\hspost}@{}}%
\column{5}{@{}>{\hspre}l<{\hspost}@{}}%
\column{8}{@{}>{\hspre}l<{\hspost}@{}}%
\column{E}{@{}>{\hspre}l<{\hspost}@{}}%
\>[3]{}\Conid{Alg-hom}\;\mathbin{:}\;\Conid{Alg}\;\to \;\Conid{Alg}\;\to \;\Conid{Type}{}\<[E]%
\\
\>[3]{}\Conid{Alg-hom}\;(\Conid{X},\Varid{θ})\;(\Conid{Y},\Varid{ρ})\;:\equiv\;{}\<[E]%
\\
\>[3]{}\hsindent{5}{}\<[8]%
\>[8]{}(\Varid{f}\;\mathbin{:}\;\Conid{X}\;\to \;\Conid{Y})\;{}\<[E]%
\\
\>[3]{}\hsindent{2}{}\<[5]%
\>[5]{}\Varid{×}\;{}\<[8]%
\>[8]{}(\Varid{f-β₀}\;\mathbin{:}\;(\Varid{x}\;\mathbin{:}\;\Conid{F}\;\Conid{X})\;\to \;\Varid{f}\;(\Varid{θ}\;\Varid{x})\;=\;\Varid{ρ}\;(\Conid{F}\;\Varid{f}\;\Varid{x})){}\<[E]%
\ColumnHook
\end{hscode}\resethooks

The witness of commutativity has suggestively been named \ensuremath{\Varid{f-β₀}} as
this gives us the \ensuremath{\Varid{β}}-rule for the recursion and induction principles.

\todoi{Something about having \ensuremath{\Varid{f₀}\;\mathbin{:}\;\Varid{f}\;\Varid{∘}\;\Varid{θ}\;\equiv\;\Varid{ρ}\;\Varid{∘}\;\Conid{F}\;\Varid{f}} instead. We need
  function extensionality either way, but this way it makes the
  arguments later on a bit easier. Also, the dependent versions don't
  work as pointfree as these ones}

\subsubsection{Homotopy initial algebras}

We call an algebra \ensuremath{(\Conid{X},\Varid{θ})} \emph{homotopy initial} if it has the
property that for every algebra \ensuremath{(\Conid{Y},\Varid{ρ})}, \ensuremath{\Conid{Alg-hom}\;(\Conid{X},\Varid{θ})\;(\Conid{Y},\Varid{ρ})}
is contractible, \ie:

\begin{hscode}\SaveRestoreHook
\column{B}{@{}>{\hspre}l<{\hspost}@{}}%
\column{3}{@{}>{\hspre}l<{\hspost}@{}}%
\column{17}{@{}>{\hspre}l<{\hspost}@{}}%
\column{21}{@{}>{\hspre}l<{\hspost}@{}}%
\column{E}{@{}>{\hspre}l<{\hspost}@{}}%
\>[3]{}\Varid{is-initial}\;\mathbin{:}\;\Conid{Alg}\;\to \;\Conid{Type}{}\<[E]%
\\
\>[3]{}\Varid{is-initial}\;\Varid{θ}\;{}\<[17]%
\>[17]{}:\equiv\;{}\<[21]%
\>[21]{}(\Varid{ρ}\;\mathbin{:}\;\Conid{Alg})\;\Varid{→}\;\Varid{is-contr}\;(\Conid{Alg-hom}\;\Varid{θ}\;\Varid{ρ})\;{}\<[E]%
\\
\>[17]{}\equiv\;{}\<[21]%
\>[21]{}(\Varid{ρ}\;\mathbin{:}\;\Conid{Alg})\;\Varid{→}\;(\Varid{f}\;\mathbin{:}\;\Conid{Alg-hom}\;\Varid{θ}\;\Varid{ρ})\;\Varid{×}\;((\Varid{g}\;\mathbin{:}\;\Conid{Alg-hom}\;\Varid{θ}\;\Varid{ρ})\;\Varid{→}\;\Varid{f}\;=\;\Varid{g}){}\<[E]%
\ColumnHook
\end{hscode}\resethooks

\subsubsection{Equality of algebra morphisms}

As we see in the definition of homotopy initiality, we need to be able
to talk about equality of algebra morphisms. Given algebras \ensuremath{(\Conid{X},\Varid{θ})}
and \ensuremath{(\Conid{Y},\Varid{ρ})} and morphisms \ensuremath{(\Varid{f},\Varid{f-β₀})} and \ensuremath{(\Varid{g},\Varid{g-β₀})} between
them, we know that, by equality on Σ-types, the following holds:

\begin{hscode}\SaveRestoreHook
\column{B}{@{}>{\hspre}l<{\hspost}@{}}%
\column{4}{@{}>{\hspre}l<{\hspost}@{}}%
\column{E}{@{}>{\hspre}l<{\hspost}@{}}%
\>[4]{}(\Varid{f},\Varid{f-β₀})\;=\;(\Varid{g},\Varid{g-β₀}){}\<[E]%
\\
\>[B]{}\Varid{≃}\;{}\<[4]%
\>[4]{}(\Varid{p}\;\mathbin{:}\;\Varid{f}\;=\;\Varid{g}){}\<[E]%
\\
\>[B]{}\Varid{×}\;{}\<[4]%
\>[4]{}(\Varid{p-β₀}\;\mathbin{:}\;\Varid{transport}\;(\Varid{λ}\;\Varid{h}\;\Varid{→}\;(\Varid{x}\;\mathbin{:}\;\Conid{F}\;\Conid{X})\;\Varid{→}\;\Varid{h}\;(\Varid{θ}\;\Varid{x})\;=\;\Varid{ρ}\;(\Conid{F}\;\Varid{h}\;\Varid{x}))\;\Varid{f-β₀}\;=\;\Varid{g-β₀}){}\<[E]%
\ColumnHook
\end{hscode}\resethooks

We not only need to show that the functions \ensuremath{\Varid{f}} and \ensuremath{\Varid{g}} are equal, but
also that their β-laws are in some sense compatible with eachother,
respecting the equality \ensuremath{\Varid{f}\;=\;\Varid{g}}.

As it turns out, the above is equivalent to something which is more
convenient in subsequent proofs:


\begin{hscode}\SaveRestoreHook
\column{B}{@{}>{\hspre}l<{\hspost}@{}}%
\column{3}{@{}>{\hspre}l<{\hspost}@{}}%
\column{4}{@{}>{\hspre}l<{\hspost}@{}}%
\column{E}{@{}>{\hspre}l<{\hspost}@{}}%
\>[4]{}\Varid{transport}\;(\Varid{λ}\;\Varid{h}\;\Varid{→}\;(\Varid{x}\;\mathbin{:}\;\Conid{F}\;\Conid{X})\;\Varid{→}\;\Varid{h}\;(\Varid{θ}\;\Varid{x})\;=\;\Varid{ρ}\;(\Conid{F}\;\Varid{h}\;\Varid{x}))\;\Varid{p}\;\Varid{f-β₀}\;=\;\Varid{g-β₀}{}\<[E]%
\\[\blanklineskip]%
\>[B]{}\Varid{≃}\;\{\mskip1.5mu \Varid{transport}\;\Varid{over}\;\Conid{Π-types}\mskip1.5mu\}\;{}\<[E]%
\\[\blanklineskip]%
\>[B]{}\hsindent{4}{}\<[4]%
\>[4]{}(\Varid{λ}\;\Varid{x}\;\Varid{→}\;\Varid{transport}\;(\Varid{λ}\;\Varid{h}\;\Varid{→}\;\Varid{h}\;(\Varid{θ}\;\Varid{x})\;=\;\Varid{ρ}\;(\Conid{F}\;\Varid{h}\;\Varid{x}))\;\Varid{p}\;(\Varid{f-β₀}\;\Varid{x}))\;=\;\Varid{g-β₀}{}\<[E]%
\\[\blanklineskip]%
\>[B]{}\Varid{≃}\;\{\mskip1.5mu \Varid{function}\;\Varid{extensionality}\mskip1.5mu\}\;{}\<[E]%
\\[\blanklineskip]%
\>[B]{}\hsindent{4}{}\<[4]%
\>[4]{}((\Varid{x}\;\mathbin{:}\;\Conid{A})\;\Varid{→}\;\Varid{transport}\;(\Varid{λ}\;\Varid{h}\;\Varid{→}\;\Varid{h}\;(\Varid{θ}\;\Varid{x})\;=\;\Varid{ρ}\;(\Conid{F}\;\Varid{h}\;\Varid{x}))\;\Varid{p}\;(\Varid{f-β₀}\;\Varid{x})\;=\;\Varid{g-β₀}\;\Varid{x}){}\<[E]%
\\[\blanklineskip]%
\>[B]{}\Varid{≃}\;\{\mskip1.5mu \Varid{transporting}\;\Varid{over}\;\Varid{equalities}\mskip1.5mu\}\;{}\<[E]%
\\[\blanklineskip]%
\>[B]{}\hsindent{4}{}\<[4]%
\>[4]{}\mathbin{!}\;(\Varid{ap}\;(\Varid{λ}\;\Varid{h}\;\Varid{→}\;\Varid{h}\;(\Varid{θ}\;\Varid{x}))\;\Varid{p})\;\Varid{∙}\;\Varid{f-β₀}\;\Varid{x}\;\Varid{∙}\;\Varid{ap}\;(\Varid{λ}\;\Varid{h}\;\Varid{→}\;\Varid{ρ}\;(\Conid{F}\;\Varid{h}\;\Varid{x}))\;\Varid{p}\;=\;\Varid{g-β₀}\;\Varid{x}{}\<[E]%
\\[\blanklineskip]%
\>[B]{}\Varid{≃}\;\{\mskip1.5mu \Varid{path}\;\Varid{algebra}\mskip1.5mu\}\;{}\<[E]%
\\[\blanklineskip]%
\>[B]{}\hsindent{3}{}\<[3]%
\>[3]{}\Varid{f-β₀}\;\Varid{x}\;\Varid{∙}\;\Varid{ap}\;(\Varid{λ}\;\Varid{h}\;\Varid{→}\;\Varid{ρ}\;(\Conid{F}\;\Varid{h}\;\Varid{x}))\;\Varid{p}\;=\;\Varid{ap}\;(\Varid{λ}\;\Varid{h}\;\Varid{→}\;\Varid{h}\;(\Varid{θ}\;\Varid{x}))\;\Varid{p}\;\Varid{∙}\;\Varid{g-β₀}\;\Varid{x}{}\<[E]%
\ColumnHook
\end{hscode}\resethooks

The last equation tells us that showing that two algebra morphisms are
equal is somewhat like giving an algebra morphism from the witness of
the β-law for \ensuremath{\Varid{f}} to that of \ensuremath{\Varid{g}}, as we can see if we draw the
corresponding diagram:

\todoi{comm diag}



\subsection{Initiality implies induction}

The induction principle tells us that for a family \ensuremath{\Conid{B}\;\mathbin{:}\;\Conid{W}\;\to \;\Conid{Type}} and
a motive \ensuremath{\Varid{m}\;\mathbin{:}\;(\Varid{x}\;\mathbin{:}\;\Conid{F}\;\Conid{W})\;\Varid{→}\;\Varid{□}\;\Conid{F}\;\Conid{B}\;\Varid{x}\;\Varid{→}\;\Conid{B}\;(\Varid{c}\;\Varid{x})}, we get a dependent
function \ensuremath{\Varid{ind}\;\mathbin{:}\;(\Varid{x}\;\mathbin{:}\;\Conid{W})\;\to \;\Conid{B}\;\Varid{x}} along with a computation rule.

Note that \ensuremath{\Varid{m}}, along with \ensuremath{\Varid{c}}, can be seen as a morphism between the
families \ensuremath{(\Conid{W},\Conid{B})} and \ensuremath{(\Conid{F}\;\Conid{W},\Varid{□}\;\Conid{F}\;\Conid{B})}.

Another way to say this is that given a function \ensuremath{\Varid{p}\;\mathbin{:}\;\Conid{B}\;\to \;\Conid{W}} for some
\ensuremath{\Conid{B}\;\mathbin{:}\;\Conid{Type}} and that we want \ensuremath{\Varid{θ}\;\mathbin{:}\;\Conid{F}\;\Conid{B}\;\to \;\Conid{B}} such that \ensuremath{\Varid{θ}}, along with
\ensuremath{\Varid{c}}, becomes a morphism \ensuremath{\Conid{Fp}\;\to \;\Varid{p}} in the arrow category. This is
equivalent to asking for an algebra \ensuremath{(\Conid{B},\Varid{θ})} along with an algebra
morphism \ensuremath{(\Conid{B},\Varid{θ})\;\to \;(\Conid{W},\Varid{c})}.



\subsection{Induction implies initiality}


\section{Homotopy initiality for restricted \onehits}



\end{document}
