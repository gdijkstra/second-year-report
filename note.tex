\documentclass[a4paper,10pt]{article}

%include paper.fmt

\usepackage{color}
\usepackage[usenames,dvipsnames,svgnames,table]{xcolor}


%% Encoding, font stuff
\usepackage[utf8]{inputenc}
\usepackage[T1]{fontenc}

%% Symbols and whatever
\usepackage{amsmath,amsthm,amssymb,stmaryrd}

%% Layout stuff and other related goodies.
\usepackage[parfill]{parskip}
\usepackage{xspace}
\usepackage{todonotes}


%% URLs and other link stuff
\usepackage{url}
\usepackage{hyperref}
\usepackage[noabbrev]{cleveref}

\hypersetup{
  colorlinks,
  citecolor=DarkBlue,
  linkcolor=black,
  urlcolor=DarkBlue}

% Bibliography stuff
\usepackage[autostyle]{csquotes}

%% Commands
\newcommand{\todoi}[1]{\todo[inline]{#1}}
\newcommand{\withoutk}{\xspace\verb+--without-K+\xspace}
\newcommand{\uip}{uniqueness of identity proofs\xspace}
\newcommand{\Uip}{Uniqueness of identity proofs\xspace}
\newcommand{\hott}{homotopy type theory\xspace}
\newcommand{\Hott}{Homotopy type theory\xspace}
\newcommand{\mltt}{Martin-L\"of's type theory\xspace}
\newcommand{\hit}{higher inductive type\xspace}
\newcommand{\Hit}{Higher inductive type\xspace}
\newcommand{\hits}{higher inductive types\xspace}
\newcommand{\Hits}{Higher inductive types\xspace}
\newcommand{\oit}{ordinary inductive type\xspace}
\newcommand{\Oit}{Ordinary inductive type\xspace}
\newcommand{\oits}{ordinary inductive types\xspace}
\newcommand{\Oits}{Ordinary inductive types\xspace}
\newcommand{\zeroconstructor}{$0$-constructor\xspace}
\newcommand{\zeroconstructors}{$0$-constructors\xspace}
\newcommand{\onehit}{$1$-HIT\xspace}
\newcommand{\onehits}{$1$-HITs\xspace}
\newcommand{\oneconstructor}{$1$-constructor\xspace}
\newcommand{\oneconstructors}{$1$-constructors\xspace}
\newcommand{\twohit}{$2$-HIT\xspace}
\newcommand{\twohits}{$2$-HITs\xspace}
\newcommand{\twoconstructor}{$2$-constructor\xspace}
\newcommand{\twoconstructors}{$2$-constructors\xspace}
\newcommand{\zerovariable}{$0$-variable\xspace}
\newcommand{\onevariable}{$1$-variable\xspace}
\newcommand{\twovariable}{$2$-variable\xspace}
\newcommand{\zerovariables}{$0$-variables\xspace}
\newcommand{\onevariables}{$1$-variables\xspace}
\newcommand{\twovariables}{$2$-variables\xspace}
\newcommand{\ie}{i.e.\xspace}
\newcommand{\eg}{e.g.\xspace}
\newcommand{\Type}{\mbox{Type}}

% Path composition from The Book.
\newcommand{\ct}{%
  \mathchoice{\mathbin{\raisebox{0.5ex}{$\displaystyle\centerdot$}}}%
             {\mathbin{\raisebox{0.5ex}{$\centerdot$}}}%
             {\mathbin{\raisebox{0.25ex}{$\scriptstyle\,\centerdot\,$}}}%
             {\mathbin{\raisebox{0.1ex}{$\scriptscriptstyle\,\centerdot\,$}}}
}

\title{Notes on \onehits}

\author{Gabe Dijkstra}

\date{}

\begin{document}

\maketitle

The data for higher inductive types are comprised of a sequence of
(higher) constructors. Each constructor can be given as a dependent
dialgebra on the category of dependent dialgebras of the previous
constructor, with $\Type$ being the base of this induction. Suppose we
have a higher inductive type $T$, given by the following constructors:

\begin{align*}
  &c_0 : (x : F_0 T)                         &\rightarrow &G_0 (T, x) \\
  &c_1 : (x : F_1 (T, c_0))                  &\rightarrow &G_1 ((T, c_0), x) \\
  &c_2 : (x : F_2 (T, c_0, c_1))             &\rightarrow &G_2 ((T, c_0, c_1), x) \\
  \vdots \\                               
  &c_k : (x : F_k (T, c_0, \hdots, c_{k-1}))  &\rightarrow &G_k ((T, c_0, \hdots, c_{k-1}), x)
\end{align*}

Where the functors have the following types:

\begin{align*}
  F_0 : \Type \rightarrow \Type \\
  F_{i+1} : (F_i,G_i)\mbox{-dialg} \rightarrow \Type \\
\end{align*}

\begin{align*}
  G_0 : \int_{\Type} F_0 \rightarrow \Type \\
  G_{i+1} : \int_{(F_i,G_i)\mbox{-dialg}} F_{i+1} \rightarrow \Type \\
\end{align*}

\todoi{Derivation of free monad approach}

\todoi{Write down type formation rules etc. for the restricted version of the \onehit.}

\end{document}